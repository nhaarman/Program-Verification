%!TEX root=PVH2QueueGroup.tex
\section{Sequential Queue implementation}

For the sequential Queue, we decided to go with a \texttt{LinkedList} implementation, which can be seen as a FIFO queue where items are appended at the tail, and taken from the head of the queue. We took the Java implementation of \texttt{LinkedList} and stripped it down to contain only code relevant to the exercise, and created a PVL implementation based on that.

The Java implementation keeps a reference to the first and last \texttt{Node} of the list as fields. This allows for easy appending and extraction of values.
We chose to create a \texttt{contents()} function which describes the current contents of the list as a \texttt{seq<int>}. That way we can easily ensure items are added and removed properly. To do that, we declared a recursive \texttt{contents()} function on \texttt{Node}. This approach requires that the first \texttt{Node} has (recursively) at least a read permission on the \texttt{val} and \texttt{next} fields of all subsequent \texttt{Nodes}, as seen in Listing \autoref{lst:Node}. This immediately leads to a problem, since we cannot easily obtain a write permission to append an item to the queue: either we have full write permissions on the contents of the \texttt{LinkedList}'s \texttt{last} field, or we have recursive read permissions for all \texttt{Nodes}. To solve this, we dropped the \texttt{last} field as a whole, and gave full recursive \emph{write} permission to the \texttt{Nodes}. Appending an item now traverses the entire list, starting from the head. The up-to-date and working version can be found in the attachments.

We had to specify the methods {\tt peek}, {\tt poll} and {\tt offer}. The latter two delegate their work to the 'private' functions {\tt unlinkFirst()} and {\tt linkLast(int)} respectively. Since we used the {\tt contents()} function in our specifications, we can easily verify and test that our implementation works as intended.

To verify the sequential implementation, run the following command:

\begin{center}
	{\tt vct Integer.pvl Node.pvl Queue.pvl Test.pvl --silver=silicon --inline}
\end{center}

\begin{figure}
\begin{lstlisting}[caption=Basic Node specification, captionpos=b]
class Node {
  Node next;
  int val;

  resource state() = Value(val) ** Value(next) ** next->state();

  requires state();
  seq<int> contents() =
  unfolding state() in (
    next == null
     ? seq<int>{val}
     : seq<int>{val} + next.contents()
  );
}
\end{lstlisting}
\label{lst:Node}
\end{figure}

